\documentclass[12pt]{beamer}

\mode<presentation> {
	\renewcommand{\familydefault}{\rmdefault}
	\usetheme{CambridgeUS}
	\usepackage[latin1]{inputenc}
	\usefonttheme{professionalfonts}
	\usepackage{times}
	\usepackage{tikz}
	\usepackage{amsmath}
	\usepackage{verbatim}
	\usepackage{enumerate}
	\usepackage{setspace}
	\usetikzlibrary{arrows,shapes}
	\usepackage{amsmath}
	\usepackage{eurosym}
	\usepackage{framed}
	\usepackage{extarrows}
}

\usepackage{graphicx}
\usepackage{booktabs}
\usepackage{url}

\title{Simulate random numbers using Halton sequence}
\author{Group 7 Project No.1}
\date[July 17$^{th}, 2016$]

\begin{document}
	
\begin{frame}
	\titlepage
\end{frame}


\begin{frame}
	\frametitle{Outline}
	\tableofcontents
\end{frame}
\section{Introduction about Halton sequence}
\begin{frame}
	\frametitle{Introduction about Halton sequence}
   \begin{itemize}
	\item  In statistics, Halton sequences are sequences used to generate points in space for numerical methods such as Monte Carlo simulations. Although these sequences are deterministic , they are of low discrepancy, that is, appear to be random for many purposes. They were first introduced in 1960 and are an example of a quasi-random number sequence.
\end{itemize}
\end{frame}

\section{Example About Halton sequence}
\begin{frame}
	\frametitle{Example About Halton sequence}
	\begin{itemize}
		\item Example of Halton sequence used to generate points in $(0,1)\times(0,1)$ in $R^{2}$ .The Halton sequence is constructed according to a deterministic method that uses coprime number as its bases. As a simple example, let us take one dimension of the Halton sequence to be based on 2 and the other on 3. To generate the sequence for 2, we start by dividing the interval $(0,1)$ in half, then in fourths, eighths, etc., which generates
$$\frac{1}{2},\frac{1}{4},\frac{3}{4},\frac{1}{8},\frac{1}{8},\frac{5}{8},\frac{3}{8},\frac{7}{8},\frac{1}{16},\frac{9}{16}\dots$$
   \end{itemize}
\end{frame}

\section{Implementation in codes in R and MATLAB}
\begin{frame}
	\frametitle{Implementation in codes in R and MATLAB}
	\begin{itemize}
		\item Our group simulate n random numbers with base b between 0 and 1 using Halton sequence , the R code are as follows
           \begin{center}
        \includegraphics[height=6cm]{tu1.png}
        \end{center}
	\end{itemize}
\end{frame}


\begin{frame}
	\frametitle{Implementation in codes in R and MATLAB}
	\begin{itemize}
        \item if $n=5,b=2$
        \begin{center}
        \includegraphics[height=6cm]{tu2.png}
        \end{center}
	\end{itemize}
\end{frame}



\begin{frame}
	\frametitle{Implementation in codes in R and MATLAB}
	\begin{itemize}
        \item The MATLAB code are as follows
        \begin{center}
        \includegraphics[height=4cm,wide=3cm]{tu3.png}
        \includegraphics[height=6cm,wide=3cm]{tu4.png}
        \end{center}
	\end{itemize}
\end{frame}


\section{Conclusion}
\begin{frame}
	\frametitle{Conclusion}
	\begin{itemize}
        \item From R, We can get the output results
        \begin{center}
        \includegraphics[height=4cm,wide=3cm]{tu5.png}
        \end{center}
     Also, In MATLAB, we can get the same results. Then we finish our simulation.
	\end{itemize}
\end{frame}




\end{document}


